\documentclass[times]{article}

\usepackage[margin=1.0in]{geometry}
\usepackage{graphicx}
\usepackage{adjustbox}
\usepackage{float}
\usepackage{placeins}
\usepackage[none]{hyphenat}
\usepackage{amsmath}
\usepackage[us]{datetime}
\usepackage[explicit]{titlesec}
\usepackage{standalone}
\usepackage{color}
\usepackage[table]{xcolor}
\usepackage{bm}
\maxdeadcycles=100000
\begin{document}
	\title{COMP SCI 5001 FS2017 Final Project}
	\author{Dalton Cole \\ drcgy5@mst.edu}
	\date{\formatdate{13}{12}{2017}}
	\maketitle
  
	For this project, Q-Learning was used to compute the optimal policy for Tinny Tim to maximize his donuty reward.

	The policy found along with the reward for each square is shown in Table \ref{tab:grid}. The configurations file used is shown in Table \ref{tab:conf}.

	As can be seen by Table \ref{tab:grid}, each of the squares where tiles fall are avoided with no arrow leading to them. Each donut square, minus the lower right hand corner, has a square pointing to the donut square. There is also only one positive square on the entire grid. This is probably because of the number of evaluations performed. If a tile hit Tim, or if he accidentally ran into a wall, he lost a reward, thus it was easier to be in the negative than positive.

	\begin{table}
		\centering
		\caption{Configuration File Used}
		\label{tab:conf}
		\begin{tabular}{| c | c |}
			\hline
			Random Seed & 0 \\
			\hline
			Discount Rate & 0.1 \\
			\hline
			TD Step Size & 0.1 \\
			\hline
			Evaluations & 123,456,789,012 \\
			\hline
		\end{tabular}
	\end{table}

	\begin{table}
\centering
\caption{Resulting Grid}
\label{tab:grid}
\begin{tabular}{| c |c |c |c |c |c |c |c |c |c |}
\hline
\cellcolor{black} & \cellcolor{black} & \cellcolor{black} & \cellcolor{black} & \cellcolor{black} & \cellcolor{black} & \cellcolor{black} & \cellcolor{black} & \cellcolor{black} & \cellcolor{black}\\
\cellcolor{black}0.0 & \cellcolor{black}0.0 & \cellcolor{black}0.0 & \cellcolor{black}0.0 & \cellcolor{black}0.0 & \cellcolor{black}0.0 & \cellcolor{black}0.0 & \cellcolor{black}0.0 & \cellcolor{black}0.0 & \cellcolor{black}0.0\\
\hline
\cellcolor{black} & \cellcolor{blue!25} $\bm{\downarrow}$ & \cellcolor{green!25} $\bm{\leftarrow}$ & \cellcolor{red!25} $\bm{\leftarrow}$ & \cellcolor{green!25} $\bm{\rightarrow$} & \cellcolor{green!25} $\bm{\rightarrow$} & \cellcolor{green!25} $\bm{\rightarrow$} & \cellcolor{green!25} $\bm{\rightarrow$} & \cellcolor{blue!25} $\bm{\leftarrow}$ & \cellcolor{black}\\
\cellcolor{black}0.0 & \cellcolor{blue!25} -0.09041 & \cellcolor{green!25} -0.7826 & \cellcolor{red!25} -0.05455 & \cellcolor{green!25} -1.139 & \cellcolor{green!25} -0.3364 & \cellcolor{green!25} -0.2377 & \cellcolor{green!25} -0.7371 & \cellcolor{blue!25} -0.207 & \cellcolor{black}0.0\\
\hline
\cellcolor{black} & \cellcolor{green!25} $\bm{\uparrow}$ & \cellcolor{black} & \cellcolor{green!25} $\bm{\downarrow}$ & \cellcolor{black} & \cellcolor{black} & \cellcolor{black} & \cellcolor{red!25} $\bm{\downarrow}$ & \cellcolor{green!25} $\bm{\uparrow}$ & \cellcolor{black}\\
\cellcolor{black}0.0 & \cellcolor{green!25} 1.018 & \cellcolor{black}0.0 & \cellcolor{green!25} -0.3946 & \cellcolor{black}0.0 & \cellcolor{black}0.0 & \cellcolor{black}0.0 & \cellcolor{red!25} -0.07519 & \cellcolor{green!25} -0.1634 & \cellcolor{black}0.0\\
\hline
\cellcolor{black} & \cellcolor{green!25} $\bm{\downarrow}$ & \cellcolor{black} & \cellcolor{green!25} $\bm{\downarrow}$ & \cellcolor{red!25} $\bm{\downarrow}$ & \cellcolor{green!25} $\bm{\downarrow}$ & \cellcolor{green!25} $\bm{\rightarrow$} & \cellcolor{green!25} $\bm{\leftarrow}$ & \cellcolor{green!25} $\bm{\uparrow}$ & \cellcolor{black}\\
\cellcolor{black}0.0 & \cellcolor{green!25} -0.1095 & \cellcolor{black}0.0 & \cellcolor{green!25} -0.24 & \cellcolor{red!25} -0.08025 & \cellcolor{green!25} -0.5231 & \cellcolor{green!25} -0.1269 & \cellcolor{green!25} -0.2535 & \cellcolor{green!25} -0.04677 & \cellcolor{black}0.0\\
\hline
\cellcolor{black} & \cellcolor{green!25} $\bm{\uparrow}$ & \cellcolor{black} & \cellcolor{green!25} $\bm{\downarrow}$ & \cellcolor{green!25} $\bm{\rightarrow$} & \cellcolor{green!25} $\bm{\downarrow}$ & \cellcolor{black} & \cellcolor{green!25} $\bm{\downarrow}$ & \cellcolor{green!25} $\bm{\downarrow}$ & \cellcolor{black}\\
\cellcolor{black}0.0 & \cellcolor{green!25} -0.177 & \cellcolor{black}0.0 & \cellcolor{green!25} -0.07754 & \cellcolor{green!25} -0.238 & \cellcolor{green!25} -0.07535 & \cellcolor{black}0.0 & \cellcolor{green!25} -0.1368 & \cellcolor{green!25} -0.0108 & \cellcolor{black}0.0\\
\hline
\cellcolor{black} & \cellcolor{green!25} $\bm{\downarrow}$ & \cellcolor{red!25} $\bm{\leftarrow}$ & \cellcolor{green!25} $\bm{\downarrow}$ & \cellcolor{green!25} $\bm{\leftarrow}$ & \cellcolor{green!25} $\bm{\leftarrow}$ & \cellcolor{black} & \cellcolor{green!25} $\bm{\uparrow}$ & \cellcolor{green!25} $\bm{\downarrow}$ & \cellcolor{black}\\
\cellcolor{black}0.0 & \cellcolor{green!25} -0.2259 & \cellcolor{red!25} -0.0845 & \cellcolor{green!25} -0.09152 & \cellcolor{green!25} -0.1432 & \cellcolor{green!25} -0.06724 & \cellcolor{black}0.0 & \cellcolor{green!25} -0.263 & \cellcolor{green!25} -0.09724 & \cellcolor{black}0.0\\
\hline
\cellcolor{black} & \cellcolor{green!25} $\bm{\downarrow}$ & \cellcolor{green!25} $\bm{\leftarrow}$ & \cellcolor{green!25} $\bm{\uparrow}$ & \cellcolor{black} & \cellcolor{red!25} $\bm{\uparrow}$ & \cellcolor{black} & \cellcolor{red!25} $\bm{\downarrow}$ & \cellcolor{green!25} $\bm{\downarrow}$ & \cellcolor{black}\\
\cellcolor{black}0.0 & \cellcolor{green!25} -0.03103 & \cellcolor{green!25} -0.8109 & \cellcolor{green!25} -0.0801 & \cellcolor{black}0.0 & \cellcolor{red!25} -0.1481 & \cellcolor{black}0.0 & \cellcolor{red!25} -0.1945 & \cellcolor{green!25} -0.127 & \cellcolor{black}0.0\\
\hline
\cellcolor{black} & \cellcolor{green!25} $\bm{\uparrow}$ & \cellcolor{red!25} $\bm{\leftarrow}$ & \cellcolor{green!25} $\bm{\downarrow}$ & \cellcolor{black} & \cellcolor{green!25} $\bm{\rightarrow$} & \cellcolor{green!25} $\bm{\downarrow}$ & \cellcolor{green!25} $\bm{\rightarrow$} & \cellcolor{green!25} $\bm{\uparrow}$ & \cellcolor{black}\\
\cellcolor{black}0.0 & \cellcolor{green!25} -0.1442 & \cellcolor{red!25} -0.003676 & \cellcolor{green!25} -0.2479 & \cellcolor{black}0.0 & \cellcolor{green!25} -0.5369 & \cellcolor{green!25} -0.02842 & \cellcolor{green!25} -0.01054 & \cellcolor{green!25} -0.05041 & \cellcolor{black}0.0\\
\hline
\cellcolor{black} & \cellcolor{blue!25} $\bm{\uparrow}$ & \cellcolor{green!25} $\bm{\leftarrow}$ & \cellcolor{green!25} $\bm{\uparrow}$ & \cellcolor{black} & \cellcolor{green!25} $\bm{\rightarrow$} & \cellcolor{green!25} $\bm{\leftarrow}$ & \cellcolor{green!25} $\bm{\leftarrow}$ & \cellcolor{blue!25} $\bm{\leftarrow}$ & \cellcolor{black}\\
\cellcolor{black}0.0 & \cellcolor{blue!25} -0.2627 & \cellcolor{green!25} -1.195 & \cellcolor{green!25} -0.2279 & \cellcolor{black}0.0 & \cellcolor{green!25} -0.2731 & \cellcolor{green!25} -0.06344 & \cellcolor{green!25} -0.07356 & \cellcolor{blue!25} -0.1539 & \cellcolor{black}0.0\\
\hline
\cellcolor{black} & \cellcolor{black} & \cellcolor{black} & \cellcolor{black} & \cellcolor{black} & \cellcolor{black} & \cellcolor{black} & \cellcolor{black} & \cellcolor{black} & \cellcolor{black}\\
\cellcolor{black}0.0 & \cellcolor{black}0.0 & \cellcolor{black}0.0 & \cellcolor{black}0.0 & \cellcolor{black}0.0 & \cellcolor{black}0.0 & \cellcolor{black}0.0 & \cellcolor{black}0.0 & \cellcolor{black}0.0 & \cellcolor{black}0.0\\
\hline
\end{tabular}
\end{table}


		
\end{document}
